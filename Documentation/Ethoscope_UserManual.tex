\documentclass[10pt,twoside, fleqn]{memoir}
\usepackage{graphicx}
\usepackage{hyperref} % HYPERLINKS
\usepackage[nomain,acronym,toc]{glossaries}
\makeglossaries
\usepackage{xcolor}
\hypersetup{
    colorlinks,
    linkcolor={blue!50!black},
    citecolor={blue!50!black},
    urlcolor={blue!80!black}
}
\definecolor{warningbackground}{RGB}{252,226,158}
\definecolor{infobackground}{RGB}{217,237,247}
\definecolor{infoforeground}{RGB}{58,135,173}
\definecolor{infoborder}{RGB}{188,232,241}
\usepackage{environ}
\usepackage{tikz}
\usetikzlibrary{fit,backgrounds,calc}

%\usepackage[T1]{fontenc}
%\usepackage{mathpazo} % USE PALATINO FONT
%\usepackage[latin1]{inputenc}
%\usepackage[xindy]{imakeidx}

%\usepackage[british]{babel}
%\usepackage[framed, numbered, autolinebreaks]{mcode}
%\usepackage{listings}

%

%\usepackage[font=footnotesize]{caption}
%\usepackage{subcaption}
%\usepackage{pdflscape} %Landscape pages
%\usepackage{pdfpages}
%\usepackage{amsmath}
%\usepackage{amssymb}
%\usepackage{gensymb} %Degree sign
%\usepackage{footnote} %Footnotes in tabulars
%\usepackage{xcolor,framed,marginnote,blindtext}
%\colorlet{shadecolor}{blue!10}
%
%
%\usepackage{tikz} %TO CREATE BLOCK DIAGRAMS
%%\usetikzlibrary{external}
%%\tikzexternalize
%%\tikzset{external/mode=graphics if exists}
%
%\usetikzlibrary{shapes,arrows}
%\usetikzlibrary{positioning}
%
%\setcounter{tocdepth}{2} % DEPTH OF TABLE OF CONTENTS; 2= SUBSECTIONS INCLUDED
%\setcounter{secnumdepth}{2} % SUBSECTIONS ARE NUMBERED
%
%\usetikzlibrary{positioning}



%% BEGIN TITLE

\makeatletter
\def\maketitle{%
  \null
  \thispagestyle{empty}%
  \vfill
  \begin{center}\leavevmode
    \normalfont
    {\LARGE\raggedleft \@author\par}%
    \hrulefill\par
    {\huge\raggedright \@title\par}%
    \vskip 1cm
%    {\Large \@date\par}%
  \end{center}%
  \vfill
  \null
  \cleardoublepage
  }
\makeatother
\author{RYMAPT}
\title{Ethoscope user manual and documentation}
\date{23 February 2015}


\newacronym{ddye}{D$_{\text{dye}}$}{donor dye, ex. Alexa 488}
\newacronym[description={\glslink{r0}{F\"{o}rster distance}}]{R0}{$R_{0}$}{F\"{o}rster distance}
\newglossaryentry{r0}{name=\glslink{R0}{\ensuremath{R_{0}}},text=F\"{o}rster distance,description={F\"{o}rster distance, where 50\% ...}, sort=R}
\newglossaryentry{kdeac}{name=\glslink{R0}{\ensuremath{k_{DEAC}}},text=$k_{DEAC}$, description={is the rate of deactivation from ... and emission)}, sort=k}

% A
\newacronym{af}{AF}{Auto Focus}
\newacronym{apt}{APT}{Advanced Positioning Technology}

% C
\newacronym{ccd}{CCD}{Charge Coupled Device}

% D
\newacronym{dio}{DIO}{Digital InputOutput}
\newacronym{dpss}{DPSS}{Diode Pump Solid State}

% F
\newacronym{fov}{FOV}{Field of View}

% G
\newacronym{gui}{GUI}{Graphic User Interface}

% I
\newacronym{ide}{IDE}{Integrated Development Environment}
\newacronym{ifd}{IFD}{Image File Directory}
\newacronym{ir}{IR}{Infra-red}

% L
\newacronym{led}{LED}{Light Emitting Diode}
\newacronym{lsm}{LSM}{Laser Sheet Microscopy}

% M
\newacronym{mosfet}{MOSFET}{Metal Oxyde Semiconductor Field-Effect Transistor }

% N
\newacronym{nas}{NAS}{Network Attached Storage}
\newacronym{ntc}{NTC}{Negative Temperature Coefficient}

% P
\newacronym{pcb}{PCB}{Printed Circuit Board}
\newacronym{pwm}{PWM}{Pulse Width Modulation}
\newacronym{psf}{PSF}{Point Spread Function}

% S
\newacronym{scl}{SCL}{Serial Clock Line}
\newacronym{sda}{SDA}{Serial DAta line}
\newacronym{shr}{SHR}{Synology Hybrid Raid}
\newacronym{svn}{SVN}{Apache SubVersion}

% T
\newacronym{tif}{TIFF}{Tagged Image File Format}
\newacronym{tps}{TPS}{Thinnest Part of the Sheet}

% U
\newacronym{usb}{USB}{Universal Serial Bus}

% R
\newacronym{roi}{ROI}{Region Of Interest} 


%%% BEGIN DOCUMENT
\makeindex
\begin{document}

\let\cleardoublepage\clearpage


\maketitle
\frontmatter

\null\vfill
\begin{center}
\begin{figure}
  \centering
  \includegraphics[width=0.80\textwidth]{./images/isometric.jpg}
  \label{fig:Intro}
\end{figure}
\end{center}
\begin{flushleft}
\textit{Ethoscope user manual and documentation}\newline
\newline
RYMAPT - 2017\newline info@rymapt.com
\bigskip

\end{flushleft}
\let\cleardoublepage\clearpage

\newpage
\tableofcontents

\mainmatter
\sloppy

\newenvironment{SpecialPar}
  {\begin{shaded}\marginnote{\fbox{NOTE}}}
  {\end{shaded}}

%%%%%%%%%%%%%%%%%%%%%%%
% COMMAND TO CREATE A WARNING BOX
\NewEnviron{alertinfo}[1]
{
    \begin{tikzpicture}
    \node[inner sep=0pt,
          draw=infoborder,
          line width=1.2pt,
          fill=infobackground] (box) {\parbox[t]{.99\textwidth}
        {%
            \begin{minipage}{.15\textwidth}
                \centering\tikz[scale=3]
                \node[scale=1]
                {
                    \includegraphics[scale=0.04]{./images/warning.png}
                };
            \end{minipage}%
           \begin{minipage}{.80\textwidth}
                \vskip 10pt
                \textbf{\textcolor{infoforeground}{#1}}\par\smallskip
                \textcolor{infoforeground}{\BODY}
                \par\smallskip
            \end{minipage}\hfill
        }%
    };

    \end{tikzpicture}
}
%%%%%%%%%%%%%%%%%%%%%%%


%%% INCLUDE CHAPTERS
%\include{ch_Introduction}
\chapter{Hardware description}\label{ch:hardware}
The Ethoscope is comprised of the following parts:
\begin{description}
  \item[Top unit] This contains the main components of the Ethoscope, such as the processing unit, the camera and the environmental monitor shield (optional). The components are enclosed in a plastic shell, as shown in figure~\ref{fig:case_top}.
    \begin{figure}
      \centering
      \includegraphics[width=0.9\textwidth]{./images/BasicComponentsTop.pdf}
      \caption{Top unit.}
      \label{fig:case_top}
    \end{figure}
      
  \item[Arena] The arena defines the experiment to be performed. The basic arena, provided with the standard kit, provides support for up to twenty glass tubes (figure~\ref{fig:basic_arena}; each tube defines a \gls{roi} within which a target can be tracked. Three black dots in the corner of the arena are used by the software to determine a system of reference.
    \begin{figure}
      \centering
      \includegraphics[width=0.8\textwidth]{./images/view_arena.pdf}
      \caption{The basic arena provided with the Ethoscope. Three black dots in the corners are used to provide alignment references to the software.}
      \label{fig:basic_arena}
    \end{figure}
    
  \item[Base unit] The unit (figure~\ref{fig:base_unit}) acts as a base for the whole Ethoscope and as a support and alignment tool for the arena. The base unit also contains the strip of \gls{led} used to provide \gls{ir} illumination to the arena. A slot in the side can be use to route wiring under the arena, for instance to install an optional vibrator unit.
    \begin{figure}
      \centering
      \includegraphics[width=0.8\textwidth]{./images/view_base.pdf}
      \caption{The basic arena provided with the Ethoscope. Three black dots in the corners are used to provide alignment references to the software.}
      \label{fig:base_unit}
    \end{figure}
      
  \item[Side walls] The walls connect the base to the top unit and ensure a correct distance between camera and arena. The standard walls are made of 3~mm acrylic sheet and are designed to ensure that the \gls{fov} of the camera covers the whole arena. 
\end{description}

\section{First setup}\label{ch:setup}
The following operations describe how to assemble and use the Ethoscope.
The top unit is shipped pre-assembled to reduce the number of operations required prior to using the Ethoscope.\\
If an additional component is purchased separately and needs to be installed, the unit can be taken apart by removing the M2.5 screw located at the bottom of the case and then opening the case to access the core electronics.
\begin{alertinfo}{WARNING}
        Be careful when disassembling and reassembling the unit. Do not force components; if in doubt contact Rymapt for assistance.\\
        The electronic components inside the Ethoscope can be damaged by \gls{esd}.\\
        Appropriate grounding is recommended whenever the electronic components are handled.
\end{alertinfo}\\

\chapter{Software description}\label{ch:software}

\section{Network Setup}\label{ch:network}
Each ethoscope unit comes pre-configured to join the WiFi network of the supplied router. Wired
access is also possible with an ethernet cable from the ethoscope to the router. The interface
for each device is available at the address printed on the case, which can be accessed in any web
browser on Windows, Mac OS X or iOS devices.

\section{Advanced use}\label{ch:advanced}
\subsection{MySQL access}\label{ch:mysql}
The recorded data can be retrieved from each ethoscope by accessing the MySQL database running on
the device. There are many open source and proprietry tools that can access MySQL databases. Read
only access is available with the username ``ethoscope'' and password ``ethoscope'' on the normal MySQL
port (3306).

\subsection{SSH access}\label{ch:mysql}
It is possible to logon to each ethoscope directly to perform custom configuration, however this is
discouraged because careless use can render the devices inoperable. In some cases this can invalidate
your warranty. For normal use it should never be necessary to access the device directly.

Advanced users who know what they're doing can access the device on the normal SSH port (22) with
the username ``pi'' and the password printed on the case of the ethoscope.

\subsection{Using a different network}\label{ch:mysql}
Ethoscope units will work on another network with no configuration if connected by an ethernet cable.
WiFi configuration is not possible with the current release of the software but will come in a
future update. It is however possible to SSH on to the device and configure other WiFi networks by
hand using the ``wpa\textunderscore{}cli'' Linux tool, but this is for advanced users only.

Please note that using Ethoscopes on open networks is discouraged since there is no encryption or
security in the current software release. Anybody on the same network can read all data and
start/stop runs. Rymapt is working to remedy this, and encrypted connections with access control
will be available in a future update along with guided WiFi setup.

%\include{ch_MainGUI}
%\include{ch_Calibration}
%\include{ch_Tracking}
%\include{ch_PSF}

%%% APPENDIX
%\include{ch_Appendix}


%%% GLOSSARY
%\printglossaries
%\printglossary[type=\acronymtype]
\printglossary[type=\acronymtype,title=Abbreviations]

%%% BIBLIOGRAPHY
%\bibliographystyle{unsrt}
%\bibliography{./../Bibliography/myBibliography}


\end{document} 